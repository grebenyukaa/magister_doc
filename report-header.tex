\documentclass[a4paper,12pt]{report}
\raggedbottom
% polyglossia should go first!
\usepackage{polyglossia} % multi-language support

\setdefaultlanguage{russian}
\setmainfont{CMU Serif}
\setsansfont{CMU Sans Serif}
\setmonofont{CMU Typewriter Text}

\setmainlanguage{russian}
\setotherlanguage{english}

\DeclareSymbolFont{letters}{\encodingdefault}{\rmdefault}{m}{it}
\usepackage{amsmath} % math symbols, new environments and stuff
%\usepackage{bm}
\usepackage{unicode-math} % for changing math font and unicode symbols
\usepackage{identfirst}
\setmathfont{XITS Math}

\parindent=1.5cm
\usepackage{indentfirst}
\usepackage[left=2cm,right=1cm,top=2cm,bottom=2cm,bindingoffset=1cm]{geometry}% 
%for margins in title page
\renewcommand{\baselinestretch}{1.5}

\usepackage[style=english]{csquotes} % fancy quoting
\usepackage{microtype} % for better font rendering
\usepackage[backend=biber, sorting=none, style=gost-numeric]{biblatex} % for 
%bibliography
\addbibresource{reference_list.bib}

\usepackage{hyperref} % for refs and URLs
\usepackage{graphicx} % for images (and title page)
\usepackage{tabu} % for tabulars (and title page)
\usepackage{placeins} % for float barriers
\usepackage{titlesec} % for section break hooks

\usepackage[labelsep=endash]{caption}
\captionsetup[table]{
    singlelinecheck=false, %table caption per GOST, not centered
    justification=justified}
\captionsetup[figure]{
    name=Рисунок           %picture caption per GOST
}

\usepackage{subcaption} % for subfloats

\usepackage{listings} % for listings
\newfontfamily\listingsfont{Source Code Pro}

\usepackage{enumitem} % for unboxed description labels (long ones)
\usepackage{tikz}     % tikz pictures
\usepackage{rotating} % landscape pictures

\defaultfontfeatures{Mapping=tex-text} % for converting "--" and "---"

\MakeOuterQuote{"} % enable auto-quotation

% new page and barrier after section, also phantom section after clearpage for
% hyperref to get right page.
% clearpage also outputs all active floats:
\newcommand{\sectionbreak}{\clearpage\phantomsection}
\newcommand{\subsectionbreak}{\FloatBarrier}
\newcommand\numberthis{\addtocounter{equation}{1}\tag{\theequation}}
\renewcommand{\thesection}{\arabic{section}} % no chapters
\numberwithin{equation}{section}

\setcounter{tocdepth}{3}

\usepackage{lastpage}
\usepackage[figure,table,xspace]{totalcount}

\usepackage{array}
\newcolumntype{L}[1]{>{\raggedright\let\newline\\\arraybackslash\hspace{0pt}}m{#1}}
\newcolumntype{C}[1]{>{\centering\let\newline\\\arraybackslash\hspace{0pt}}m{#1}}
\newcolumntype{R}[1]{>{\raggedleft\let\newline\\\arraybackslash\hspace{0pt}}m{#1}}

\makeatletter
\define@key{blx@lbx}{fromjapanese}{\blx@defstring{fromjapanese}{#1}}
\define@key{blx@lbx}{langjapanese}{\blx@defstring{langjapanese}{#1}}
\makeatother

%algorithm2e
\usepackage{algorithm2e}
\SetKwInput{KwData}{Исходные параметры}
\SetKwInput{KwResult}{Результат}
\SetKwInput{KwIn}{Входные данные}
\SetKwInput{KwOut}{Выходные данные}
\SetKwIF{If}{ElseIf}{Else}{если}{тогда}{иначе если}{иначе}{конец условия}
\SetKwFor{While}{до тех пор, пока}{выполнять}{конец цикла}
\SetKw{KwTo}{от}
\SetKw{KwRet}{возвратить}
\SetKw{Return}{возвратить}
\SetKwBlock{Begin}{начало блока}{конец блока}
\SetKwSwitch{Switch}{Case}{Other}{Проверить значение}{и выполнить}{вариант}{в противном случае}{конец варианта}{конец проверки значений}
\SetKwFor{For}{цикл}{выполнять}{конец цикла}
\SetKwFor{ForEach}{для каждого}{выполнять}{конец цикла}
\SetKwRepeat{Repeat}{повторять}{до тех пор, пока}
\SetAlgorithmName{Рисунок}{рисунок}{Список алгоритмов}

